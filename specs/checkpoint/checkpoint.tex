\documentclass{article}
\usepackage{chicken}
\usepackage{verbatim}

\DateAndTitle{17 April 2015}{Final Specification}

\usepackage{xcolor}
\newcommand{\todo}[1]{\textcolor{red}{\textbf{\textsf{[#1]}}}}

\begin{document}


\section{Proposal Revisions}
\subsection{Overview}
\begin{description}
  \item[\textbf{Problem:}]
    Using Coq, verify Timsort, Python's preferred sorting algorithm!
Yet, implementations of Timsort --
a hybrid of insertion sort and
mergesort, plus some optimizations using heuristics about memory management --
have been known
to fail in subtle ways.
We turn to Coq for its support of \emph{verifiable} code and draw on
its formal proof techniques, hoping to implement and verify (a version of)
Timsort.
Additionally, Coq supports the
exporting of programs into other languages such as ocaml, so we expect to
render a sorting procedure that is not only verified, but also
usable in a broader context.

  \item[\textbf{Solution sketch:}]
    We will take an incremental approach (and all of our algorithms
    will be functional, using persistent data structures).
    We're going to start with a
    library of fundamental data structures and verified methods.
    Then, we'll work toward verifying a
    simplified version of Timsort,
    hybridizing mergesort and insertion sort
    with a small subset of the heuristics used by Timsort in full.
    At first, all of these components of this simplified Timsort will
    be independently verified; the combined algorithm implementation
    will be thoroughly unit-tested.
    For short, call this algorithm Simsort.
    Implementing Simsort will be the conclusion of our core functionality.

  \item[\textbf{Goals:}]
    Primarily, we'd like to verify Timsort (i.e., Simsort) as a way to
    learn more about Coq and certified programming.
    \textbf{First, for our core functionality, we'll work through the first five
    chapters of \emph{Software Foundations} \cite{sf},
    producing a cohesive set of
    solutions while also building up the foundational data structures
    necessary for implementation and verification of our chosen
    sorting algorithms.}
\end{description}


\subsection{Prioritized Feature List}
\emph{Note:} All algorithms and data structures used in this project will
be functional; in particular, we'll use persistent data structures.

\paragraph{Core Features.}
The following feature list is a minimal list of the chapters of \emph{SF}
for which we will solve the exercises and refactor the code from the solutions
into libraries to be used for our later verification of sorting algorithms.
\begin{enumerate}
  \item \textbf{Basics.}
    Enumerated types: Booleans, Function types, Peano naturals.
    Proofs by simplification, rewriting, and case analysis.
    Notations.
  \item \textbf{Induction.}
    Proofs by induction.
    Using lemmas.
  \item \textbf{Lists.}
    Tuples of naturals.
    Lists of naturals.
    Variant and record types.
    Proofs by induction on lists.
  \item \textbf{Polymorphism.}
    Polymorphic lists, functions, tuples, and variants.
    Essential function methods (\verb`map`, \verb`fold`, etc.).
    Proofs using the \verb`unfold` tactic.
  \item \textbf{More Coq.}
    More on induction hypotheses.
    Proof tactics \verb`apply`, \verb`apply with`,
    \verb`inversion`, \verb`destruct`.
\end{enumerate}

\paragraph{Cool Extensions}
\begin{enumerate}
  \item \textbf{Verified insertion sort.}
    Verified insertion sort of lists of natural numbers.
  \item \textbf{Verified merge sort.}
    Verified merge sort of lists of natural numbers.
  \item \textbf{Simsort.}
    Fully tested implementation of Simsort (our hybridization of verified
    merge sort, verified insertion sort, and a modified subset of the
    heuristics used in Timsort).
    We still need to determine exactly which subset of heuristics we will
    use and how we will modify them.
  \item \textbf{Verified Simsort.}
    This is our main goal beyond core functionality.
    We will improve the fully tested Simsort to
    a rigorously verified Simsort (using Coq).
\end{enumerate}

\paragraph{Cooler Extensions}
\begin{enumerate}
  \item \textbf{Heaps.}
    Polymorphic priority queues.
    If it provides an advantage in asymptotics,
    we will use heaps to re-implement stacks.
  \item \textbf{Heap Sort.}
    Verified heap sort of heaps of natural numbers.
    This will operate on lists of natural numbers, represented perhaps
    as trees or priority queues.
  \item \textbf{Augmenting Simsort with heapsort.}
    Verified implementation of Simsort, with heapsort replacing
    insertion sort, for a slight improvement in asymptotics.
  \item \textbf{Passing foreign tests.}
    We have come across a few known to be broken implementations of Timsort
    in certain languages (e.g., Java's clone of Timsort,
    early versions of Python 2.x's Timsort).
    For this cool extension, we would take some of the failing test cases for
    those other implementations, adapt them to use the same heuristic
    assumptions that we've used with Simsort, and show that our verified
    Simsort passes those tests.
  \item \textbf{Adding more heuristics.}
    If we make it this far, we will add more heuristics to Simsort,
    showing that each addition passes verification and doesn't break
    invariants, working our way gradually to a verified, functional
    Timsort in full.
\end{enumerate}


\section{Progress Report}
\subsection{Progress}
  So far, we have worked out most of the exercises from the first five chapters.
  Our implementations work well, though they have yet to be refactored into
  modules for use by our proofs of sorting algorithms.
\subsection{Problems}
  We had concerns at the beginning of the week regarding our original plan to
  launch right into verifying sorting algorithms without first builing a cohesive
  code library by completing the necessary early exercises from \emph{SF}.
  We have since spoken to Prof. Morissett and to Ore about including this work
  as the core functionality part of our project, rather than simply as
  background, and we feel that this is now a more tenable project.
\subsection{Teamwork}
  So far, teamwork has been pretty good. We've had a little trouble reconciling
  schedules and disparate project goals over the last week, but it's fair to
  say that, with the project revision as outlined above, we feel more on the
  same page. Early on, Michael and Ben were more available for the early project
  planning. Availabilities have fluctuated over the past week, so Alex and
  Michael now have some catching up to do with learning the Coq basics,
  but we feel confident that we will be on the same page by the end of the
  weekend, ready to tackle the sorting algorithms cohesively.
  Morale seems good.
\subsection{Plan}
  We plan on finishing the remaining exercises and refactoring this weekend,
  possibly Monday.
  Then, we hope to have the two basic sorting algorithms implemented and verified
  by midweek.
  Then, we will make as much progress as possible toward verifying Simsort
  as time allows on Thursday and Friday.

  We will continue our thus-far effective strategy of pair programming,
  possibly working all-together to design sorting algorithm proofs.


\begin{thebibliography}{xx}
  \bibitem{cpdt}
    Chlipala, Adam. \emph{Certified Programming with Dependent Types}.

  \bibitem{sf}
    Pierce, Benjamin, et al. \emph{Software Foundations}.

\end{thebibliography}



\end{document}
