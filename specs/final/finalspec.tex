\documentclass{article}
\usepackage{chicken}

\DateAndTitle{17 April 2015}{Final Specification}

\usepackage{xcolor}
\newcommand{\todo}[1]{\textcolor{red}{\textbf{\textsf{[#1]}}}}

\begin{document}


\section{Names and Emails}
\begin{itemize}
  \item Aleksandar Makelov --- \texttt{amakelov@college.harvard.edu}
  \item Ben Wetherfield --- \texttt{bwetherfield@college.harvard.edu}
  \item Chan Kang --- \texttt{chankang@college.harvard.edu}
  \item Michael Fountaine --- \texttt{mfount@college.harvard.edu}
\end{itemize}


\section{Overview}
\begin{description}
  \item[\textbf{Problem:}]
    Using Coq, verify Timsort, python's preferred sorting algorithm!
    \todo{Improve motivation? Possibly include discussion of failures of
    implementations of Timsort in languages other than Python 3.x.}
    Timsort is a hybrid of insertion sort and mergesort, plus some heuristics
    about memory management and other optimizations.
    Our simplified version will include a reduced version of these
    heuristics.

  \item[\textbf{Solution sketch:}]
    We will take an incremental approach (and all of our algorithms
    will be functional, using persistent data structures).
    We're going to start with a
    simplified version of Timsort,
    hybridizing mergesort and insertion sort
    with a small subset of the heuristics used by Timsort in full.
    At first, all of these components of this simplified Timsort will
    be independently verified; the combined algorithm implementation
    will be thoroughly unit-tested.
    For short, call this algorithm Simsort.
    Implementing Simsort will be the conclusion of our core functionality.

    Next, we will implement heaps and a verified heapsort;
    by replacing insertion sort with heapsort in Simsort,
    we should get a constant-factor time improvement.
    Then, our primary goal beyond core functionality will
    be verification of Simsort.
    From there, we will implement extra extensions, as discussed below,
    possibly adding more heuristics to Simsort, approaching verification
    of Timsort in full.

  \item[\textbf{Goals:}]
    Primarily, we'd like to verify Timsort (i.e., Simsort) as a way to
    learn more about Coq and certified programming.

\end{description}


\section{Prioritized Feature List}
\emph{Note:} All algorithms and data structures used in this project will
be functional; in particular, we'll use persistent data structures.

\paragraph{Core Features}
\begin{itemize}
  \item \textbf{Fundamentals.}
    Booleans, natural numbers (defined inductively), polymorphic lists,
    stacks (for very basic representations of memory needed within Timsort
    heuristics).
  \item \textbf{Verified insertion sort.}
    Verified insertion sort of lists of natural numbers.
  \item \textbf{Verified merge sort.}
    Verified merge sort of lists of natural numbers.
  \item \textbf{Simsort.}
    Fully tested implementation of Simsort (our hybridization of verified
    merge sort, verified insertion sort, and a modified subset of the
    heuristics used in Timsort).
    \todo{talk about what the heuristics might be}

\end{itemize}

\paragraph{Cool Extensions}
\begin{itemize}
  \item \textbf{Heaps.}
    Priority queues of natural numbers.
  \item \textbf{Heap Sort.}
    Verified heap sort of heaps of natural numbers.
    This will operate on lists of natural numbers, represented perhaps
    as trees or priority queues.
  \item \textbf{Augmenting Simsort with heapsort.}
    Fully tested implementation of Simsort, with heapsort replacing
    insertion sort, for a slight improvement in asymptotics.
  \item \textbf{Verified Simsort.}
    This is our main goal beyond core functionality.
    We will improve the fully tested Simsort to
    a rigorously verified Simsort (using Coq).
  \item \textbf{Passing foreign tests.}
    We have come across a few known to be broken implementations of Timsort
    in certain languages (e.g., Java's clone of Timsort,
    early versions of Python 2.x's Timsort).
    For this cool extension, we would take some of the failing test cases for
    those other implementations, adapt them to use the same heuristic
    assumptions that we've used with Simsort, and show that our verified
    Simsort passes those tests.
  \item \textbf{Adding more heuristics.}
    If we make it this far, we will add more heuristics to Simsort,
    showing that each addition passes verification and doesn't break
    invariants, working our way gradually to a verified, functional
    Timsort in full.
\end{itemize}


\section{Technical Specification}
\paragraph{Data Structures}
We will, of course, prove all methods associated with these types/structures.

\begin{itemize}
  \item Ordered set.
  \item Natural numbers.
  \item List of natural numbers.
  \item Array of natural numbers.
    (Unclear if necessary.)
  \item Stack.
  \item Tree.
  \item Heap (a.k.a. Priority Queue).
\end{itemize}

\paragraph{Algorithms}
These are the algorithms and subroutines we'll need for the mutable (and often
in-place) versions of the sorts, which we expect to be harder; if we decide to
do the pure versions, we will make adjustments accordingly. The ideas are based
on the paper \cite{filliatre1999certification}, which describes how to prove
correctness for insertion sort, quicksort and mergesort. It also provides some
general predicates for formally verifying sorting algorithms, like
\verb|(sorted t)| which tests for sortedness, and \verb|(permutation t t')|
which tests whether $t'$ is a permutation of $t$.
\begin{itemize}
  \item
\textbf{Insertion sort:} The subroutine \verb|(insertion n t)| takes as input
an array $t$ where the first $n-1$ elements are sorted, and returns an array
where the first $n$ elements are sorted, by inserting the $n$-th element in its
place. Then insertion sort itself proceeds by running \verb|(insertion n t)|
for $n=1,2,3,\ldots,N$ for a list $t$ of length $N$.
  \item
\textbf{Mergesort:} Here the basic subroutine is \verb|(merge t)| which takes a
list $t$ in which the first and second halves are sorted, and merges them. Here
we'll need to use some extra memory. Then mergesort proceeds by calling itself
recursively on each half, and then merging them together.
  \item
\textbf{Tree sort:} We first build a binary search tree from the array by a
subroutine \verb|(bst_build t)|; then we have another subroutine
\verb|(bst_traverse t)| that traverses it in-order.
  \item
\textbf{Heap sort:} The algorithm proceeds by making the array into a heap
in-place, and then gradually pushing the largest elements to the right side of
the heap. Here, we're representing a binary tree implicitly as an array, where
the children of the $i$-th element are the $2i+1$-th and $2i+2$-th elements. We
have the following subroutines:
  \begin{itemize}
  \item
Predicate \verb|(heap t n k)| checks if in a list $t$, the tree of elements of
index $\leq n$ rooted at the $k$-th element.
  \item
Predicate \verb|(inftree t n v k)| checks if in a list $t$, every element in
the tree of elements of index $\leq n$ rooted at the $k$-th element is less
than $v$.
  \item
Subroutine \verb|(downheap t k n)| takes a list $t$ where the tree of elements
of index $\leq n$ rooted at the $k$-th element is a heap \emph{except} possibly
for the root node. It then makes a bunch of swaps that make this tree into a
heap.
\end{itemize}
Having these subroutines, heapsort is easy: first build the heap using
\verb|downheap| a bunch of times, and then swap the 1-st element with the
$i$-th and rebuild the heap up to index $i-1$, for $i = N, N-1, \ldots, 1$
  \item
\textbf{Simple timsort:} The idea of timsort is to use the sorted chunks
(called \emph{runs}) often present in real-world data. At a high level, the
algorithm works by first making sure all runs are of some minimum length (using
insertion sort if necessary), and then passing over the list and merging the
runs in an intelligent way.

  As we do the second pass, we push the starting index and length of each run
on a stack, and merge consecutive runs in order to preserve a certain invariant
of the stack:  if $A,B,C$ are the lengths of three consecutive runs on the
stack (with $A$ being the topmost), we require that $B>C$ and $A>B+C$.

  So at any point we end up with a bunch of runs whose sizes grow faster than
the Fibonacci sequence, i.e. at least exponentially fast; so it's easy to see
there are at most logarithmically many runs at each point in time, which is
important for memory reasons and for running time (it's faster to merge a
sequence of lists with exponentially-increasing size than to naively merge a
list split into equal parts).

  The way the invariant is preserved is where things get weird: apparently,
there is a bug in the algorithm on the wikipedia page (which attempts to
restore the invariant by only fixing the top three runs), but there is a fix
\cite{deopenjdk} we have to look into.

  It's hard to know in advance what interface would be best for our proof, but
here's a guess:
\begin{itemize}
  \item Predicate \verb|(run t i j)| - this checks if the subarray between
the $i$-th and $j$-th index of $t$ is a run, i.e. is already in sorted order.
  \item Subroutine \verb|insertion-sort| - this is just as insertion sort above.
  \item Subroutine \verb|merge| - this is supposed to take the next run and do
any merges necessary to preserve the invariant; of course, it will be based on
the merge procedure.
  \item Now piecing together all the parts should be easy. In the end, we end
up with a sequence of runs that we merge in the obvious way.
\end{itemize}

  \item \textbf{Timsort:}
  Additionally, there are some memory optimizations (`galloping') during the
merges that we might try to include if we end up having the time for it.
\end{itemize}

\paragraph{Proofs}  % A note on them....
\emph{Note:} We will be able to better characterize proofs after
working through \emph{Software Foundations} \cite{sf}
(See ``Next Steps'' below.)

At this stage, we know:
\begin{itemize}
\item
  Because Coq proofs are very formal, they suffer from an inability to
  convey human-to-human understanding that is afforded by
  semi-formal, traditional mathematical proof.
  To aid debugging and authoring of proofs in Coq,
  \emph{SF} \cite{sf} recommends two strategies, ``Informal Proof''
  and ``Human Proof'', that by encouraging verbosity and a high degree
  of refactoring
  help to restore the ``understanding'' aspect of a proof.
\item Proofs in coq are guided by tactics. These are like hints you
  give to the prover, such as ``rewrite this as the other side of this
  equality we already proved'', ``rewrite both sides in a canonical
  way'',  etc.
\item We expect to use the ``Reflexivity'' keyword frequently for
  verifying more
  basic properties of our data structures.
  Proofs in Coq that use reflexivity as a tactic are generally very
  simple proofs using, say, rewrite rules.
\item When we get onto proofs of correctness of algorithms, we will
  have to break proofs into subcomponents.
\item There are various ways of doing this:
\begin{enumerate}
\item Example, Theorem, Lemma, Fact and Remark keywords (all
  functionally the same) break up larger proof into subproofs.
\item The Case proof tactic (chapter 3 of Software Foundations) can be
  used to break a proof clearly into cases (as in the familiar
  structure of an induction proof).
\item This could break a proof into say a case of the Reflexivity
  tactic and the Induction tactic.
  Our reading of the references and some example Coq code
  indicates that this is a common formulation that appears
  when doing structural induction.
\item Ltac can be used to create common tactics.
\end{enumerate}

\end{itemize}

\section{Next Steps}
\todo{Revise these!}
\begin{itemize}
  \item
    Install Coq 8.4-pl5 and ensure that it's running correctly.
    Also install an IDE, most likely emacs with Proof General.
  \item
    Read early parts of \emph{SF} \cite{sf} and \emph{CPDT} \cite{cpdt},
    doing excercises to familiarize ourselves with Gallina (Coq's functional
    language, similar to Caml) syntax and Coq's Proof functionality.

    Specifically, we will initially work through at least the 1-star and
    2-star exercises in the opening three chapters of \emph{SF}.
    (This will give us the implementations of lists and nats.)

  \item
    In a similar vein, get an idea of how long it would take to
    verify correctness of the basic sorting algorithms

  \item
    Understand timsort in more depth.
\end{itemize}

\begin{thebibliography}{xx}
  \bibitem{cpdt}
    Chlipala, Adam. \emph{Certified Programming with Dependent Types}.

  \bibitem{sf}
    Pierce, Benjamin, et al. \emph{Software Foundations}.

  \bibitem{filliatre1999certification}
    Filliatre, Jean-Christophe and Magaud, Nicolas. \emph{Certification of
    sorting algorithms in the system COQ}

  \bibitem{deopenjdk}
    de Gouw, Stijn and Rot, Jurriaan and de Boer, Frank S and Bubel, Richard
    and Hahnle, Reiner. \emph{OpenJDK's java. utils. Collection. sort () is
    broken: The good, the bad and the worst case}
\end{thebibliography}



\end{document}
