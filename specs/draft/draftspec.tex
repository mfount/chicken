\documentclass{article}
\usepackage{chicken}

\DateAndTitle{10 April 2015}{Draft Specification}

\begin{document}

\section{Names and Emails}
\begin{itemize}
  \item Aleksandar Makelov --- \texttt{amakelov@college.harvard.edu}
  \item Ben Wetherfield --- \texttt{bwetherfield@college.harvard.edu}
  \item Chan Kang --- \texttt{chankang@college.harvard.edu}
  \item Michael Fountaine --- \texttt{mfount@college.harvard.edu}
\end{itemize}

\section{Overview}
\begin{description}
  \item[\textbf{Problem:}]
    Verify Timsort, python's preferred sorting algorithm!
  \item[\textbf{Solution sketch:}]
    We will take an incremental approach:
    We're going to start with an abstract
    "implementation" of Timsort,
    possibly black-boxing certian difficult-to-implement features
    such as heuristics used in the algorithm,
    and implement a verified version of that implementation to sort
    lists of natural numbers (defined inductively).
    Timsort is a hybrid sorting
    algorithm, requiring other sorting sub-algorithms and some simple data
    structures, so we will divide our project into interfaces for each of
    these, along with interfaces for proof tactics.
  \item[\textbf{Goals:}]
    Primarily, we'd like to verify Timsort as a way to learn more about
    Coq and certified programming. (Or, if Timsort proves to be too
    complicated (by our estimation by the time the final spec is due),
    we hope to verify another hybrid/adaptive sorting algorithm,
    with that same learning outcome in mind.)
\end{description}

\section{Prioritized Feature List}
\paragraph{Core Features}
\paragraph{Cool Extensions}

\section{Technical Specification}
\section{Next Steps}

\end{document}
