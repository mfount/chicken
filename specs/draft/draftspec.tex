\documentclass{article}
\usepackage{chicken}

\DateAndTitle{10 April 2015}{Draft Specification}

\begin{document}


\section{Names and Emails}
\begin{itemize}
  \item Aleksandar Makelov --- \texttt{amakelov@college.harvard.edu}
  \item Ben Wetherfield --- \texttt{bwetherfield@college.harvard.edu}
  \item Chan Kang --- \texttt{chankang@college.harvard.edu}
  \item Michael Fountaine --- \texttt{mfount@college.harvard.edu}
\end{itemize}


\section{Overview}
\begin{description}
  \item[\textbf{Problem:}]
    Using Coq, verify Timsort, python's preferred sorting algorithm!
  \item[\textbf{Solution sketch:}]
    We will take an incremental approach:
    We're going to start with an abstract
    "implementation" of Timsort,
    possibly black-boxing certian difficult-to-implement features
    such as heuristics used in the algorithm,
    and implement a verified version of that implementation to sort
    lists of natural numbers (defined inductively).
    Timsort is a hybrid sorting
    algorithm, requiring other sorting sub-algorithms and some simple data
    structures, so we will divide our project into interfaces for each of
    these, along with interfaces for proof tactics.
    We're going to hone our sorting algorithm verification chops
    by working through insertion sort and mergesort, both of which
    are part of Timsort, followed by tree sort and heap sort
    to test our data structures, followed by Timsort itself!
  \item[\textbf{Goals:}]
    Primarily, we'd like to verify Timsort as a way to learn more about
    Coq and certified programming. (If Timsort proves to be too
    complicated (by our estimation by the time the final spec is due),
    we will be able to accomplish this same learning outcome by working
    through insertion, merge, tree, and heap sorts.)
\end{description}

% A) fundamentals of coq & proofs -> prove mergesort -> prove insertion sort -> prove a simplified version of timsort with black boxes around the heuristics -> (now into bonus...) prove timsort with heuristics -> prove python's timsort -> ...


\section{Prioritized Feature List}
(\emph{Note:} Generally, we need to read more about Coq to understand
where to draw the line between ``core'' and ``cool''; that is, how long
these features will take to implement, including proofs.)

\paragraph{Core Features}
\begin{itemize}
  \item \textbf{Fundamentals.}
    Ordered sets, natural numbers (defined inductively), lists and arrays,
    stacks (to be used as Timsort's memory and temporary memory).
  \item \textbf{Insertion sort.}
    Verified insertion sort of lists of natural numbers.
  \item \textbf{Merge sort.}
    Verified merge sort of lists of natural numbers.
  \item \textbf{Trees.}
    Binary search trees of natural numbers.
  \item \textbf{Tree Sort.}
    Verified tree sort of
    binary search trees of natural numbers.
  \item \textbf{Heaps.}
    Priority queues of natural numbers.
  \item \textbf{Heap Sort.}
    Verified heap sort of
    binary search trees of natural numbers.
  \item \textbf{Simplified Timsort.}
    This will operate on lists of natural numbers, represented perhaps
    as trees or priority queues.

    Timsort is a hybrid of insertion sort and mergesort, plus some heuristics
    about memory management and other optimizations. Our ``simplified
    timsort'' will omit these heuristics when possible.
\end{itemize}

\paragraph{Cool Extensions}
\begin{itemize}
  \item \textbf{Timsort.}
    Verified timsort; that is, simplified timsort plus heuristics.
    This version would have the same time asymptotics as python's
    implementation of Timsort:
    $\Theta(n)$ best case,
    $\Theta(n \log n)$ average case,
    $\Theta(n \log n)$ worst case,
    where $n$ is the length of the list.

  \item \textbf{Polymorphic timsort.}
    A verified timsort that works on polymorphic lists. (Unlike in OCaml,
    this seems actually to be a good bit of work in Coq, having to prove
    things about each subtype of our ordered set type.)

  \item \textbf{Export to OCaml.}
    One of Coq's initial purposes is to export verified OCaml (or Haskell
    or Scheme) code; for this cool extension, we can try to turn some or all
    of our verified algorithm implementations into usable OCaml.

  \item \textbf{Python's timsort.}
    This final possible extension could be to modify our verified Timsort
    algorithm to use exactly the heuristics found in the current Python 3.x
    release, as opposed to the heuristics we end up using in our implemented
    version of the full Timsort algorithm.
    We could also attempt to use the same space complexity of
    Timsort, $O(n)$.
\end{itemize}


\section{Technical Specification}
\paragraph{Data Structures}
\paragraph{Algorithms}
\begin{itemize}
  \item \textbf{Mergesort:}
  \item \textbf{Insertion sort:}
  \item \textbf{Tree sort:}
  \item \textbf{Heap sort:}
  \item \textbf{Simple timsort:}
    Timsort, as described below, omitting heuristics.
  \item \textbf{Timsort:}
	  First we pass over the list and make sure each run is of at
    least some minimum length $c$.

    Then we pass over it again and push the base address (that is, the index of
    the first element) and length of every run (this could be done in the above
    pass, but let's separate them for clarity).
    But as we push runs on the stack,
    we also sometimes merge consecutive runs until some invariant
    (that `attempts
    to keep the run lengths as close to each other as possible to balance the
    merges' as the wikipedia page says) is satisfied.
    The condition is that if
    $X,Y,Z$ are the lengths of the top three runs on the stack, we must have
	  \begin{align*}
	  X > Y + Z \text{ and } Y >Z
    \end{align*}
    (There are also other invariants to maintain.)
	  So at any point we end up with a bunch of runs whose sizes grow
    faster than the Fibonacci sequence, i.e. at least exponentially fast; so
    it's easy to see there are at most logarithmically
    many runs at each point in
    time, which seems to be important for memory reasons.
    It's also important for
    running-time reasons it seems - it's much faster to merge these
    exponentially-increasing guys than to naively merge a list split into
    equal parts!
	  Additionally, there are some memory optimizations (`galloping').
\end{itemize}

\paragraph{Proofs}  % A note on them....

\section{Next Steps}
\begin{itemize}
  \item
    Install Coq 8.4-pl5 and ensure that it's running correctly.
    Also install an IDE, most likely emacs with Proof General.
  \item
    Read early parts of \emph{SF} \cite{sf} and \emph{CPDT} \cite{cpdt},
    doing excercises to familiarize ourselves with Gallina (Coq's functional
    language, similar to Caml) syntax and Coq's Proof functionality.

    Specifically, we will initially work through at least the 1-star and
    2-star exercises in the opening three chapters of \emph{SF}.
    (This will give us the implementations of lists and nats.)
\end{itemize}

\begin{thebibliography}{xx}
  \bibitem{cpdt}
    Chlipala, Adam. \emph{Certified Programming with Dependent Types}.

  \bibitem{sf}
    Pierce, Benjamin, et al. \emph{Software Foundations}.
\end{thebibliography}

\end{document}
